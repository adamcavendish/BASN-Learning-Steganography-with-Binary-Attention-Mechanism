\abstract{%
Secret information sharing through image carrier has aroused much research attention in recent years with images' growing domination on the Internet and mobile applications. The technique of embedding secret information in images without being detected is called image steganography. With the booming trend of convolutional neural networks (CNN), neural-network-automated tasks have empowered more deeply in our daily lives. However, a series of wrong labeling or bad captioning on the embedded images leaves a trace of skepticism and finally leads to a self-confession alike exposure. To improve the security of image steganography and minimize task result distortion, models must maintain the feature maps generated by task-specific networks being irrelative to any hidden information embedded in the carrier. This paper introduces a binary attention mechanism into image steganography to help alleviate the security issue, and in the meanwhile, increase embedding payload capacity. The experimental results show that our method has the advantage of high payload capacity with little feature map distortion and still resist detection by state-of-the-art image steganalysis algorithms.%
}

\keyword{convolutional neural network; steganography; attention mechanism}
